\section{Introduction}

\aspect{} --- short for Advanced Solver for Problems in Earth's ConvecTion ---
is a code intended to solve the equations that describe thermally driven
convection with a focus on doing so in the context of convection in the earth
mantle. It is primarily developed by computational scientists at Texas A\&M
University based on the following principles:
\begin{itemize}
\item \textit{Usability and extensibility:} Simulating mantle convection is a
  difficult problem characterized not only by complicated and nonlinear
  material models but, more generally, by a lack of understanding which parts
  of a much more complicated model are really necessary to simulate the
  defining features of the problem. To name just a few examples:
  \begin{itemize}
  \item Mantle convection is often solved in a spherical shell geometry, but
    the earth is not a sphere -- its true shape on the longest length scales is
    dominated by polar oblateness, but deviations from spherical shape
    relevant to convection patterns may go down to the length scales of
    mountain belts, mid-ocean ridges or subduction trenches. Furthermore,
    processes outside the mantle like crustal depression during glaciations
    can change the geometry as well.
  \item Rocks in the mantle flow on long time scales, but on shorter time
    scales they behave more like a visco-elasto-plastic material as they break
    and as their crystalline structure heals again. The mathematical models
    discussed in Section~\ref{sec:models} can therefore only be
    approximations.
    \item If pressures are low and temperatures high enough, rocks melt,
      leading to all sorts of new and interesting behavior.
  \end{itemize}
  This uncertainty in what problem one actually wants to solve requires a code
  that is easy to extend by users to support the community in determining what
  the essential features of convection in the earth mantle are. Achieving this
  goal also opens up possibilities outside the original scope, such as the
  simulation of convection in exoplanets or the icy satellites of the gas
  giant planets in our solar system.

\item \textit{Modern numerical methods:} We build \aspect{} on numerical
  methods that are at the forefront of research in all areas -- adaptive mesh
  refinement, linear and nonlinear solvers, stabilization of
  transport-dominated processes. This implies complexity in our algorithms,
  but also guarantees highly accurate solutions while remaining efficient in
  the number of unknowns and with CPU and memory resources.

\item \textit{Parallelism:} Many convection processes of interest are
  characterized by small features in large domains -- for example, mantle
  plumes of a few tens of kilometers diameter in a mantle almost 3,000 km
  deep. Such problems can not be solved on a single computer but require
  dozens or hundreds of processors to work together. \aspect{} is designed
  from the start to support this level of parallelism.

\item \textit{Building on others' work:} Building a code that satisfies above
  criteria from scratch would likely require several 100,000 lines of
  code. This is outside what any one group can achieve on academic time
  scales. Fortunately, most of the functionality we need is already available
  in the form of widely used, actively maintained, and well tested and
  documented libraries, and we leverage these to make \aspect{} a much smaller
  and easier to understand system. Specifically, \aspect{} builds immediately
  on top of the \dealii{} library (see \url{https://www.dealii.org/}) for
  everything that has to do with finite elements, geometries, meshes, etc.;
  and, through \dealii{} on Trilinos (see \url{http://trilinos.org/})
  for parallel linear algebra and on \pfrst{} (see
  \url{http://www.p4est.org/}) for parallel mesh handling.

\item \textit{Community:} We believe that a large project like \aspect{} can
  only be successful as a community project. Every contribution is welcome and
  we want to help you so we can improve \aspect{} together.

\end{itemize}

Combining all of these aspects into one code makes for an interesting
challenge. We hope to have achieved our goal of providing a useful tool to the
geodynamics community and beyond!


\note{\aspect{} is a community project. As such, we encourage contributions
  from the community to improve this code over time. Natural candidates for
  such contributions are implementations of new plugins as discussed in
  Section~\ref{sec:plugins-concrete} since they are typically self-contained and do not
  require much knowledge of the details of the remaining code. Obviously,
  however, we also encourage contributions to the core functionality in any
  form! If you have something that might be of general interest, please
  contact us.}

\note{\aspect{} will only solve problems relevant to the community if we get
  feedback from the community on things that are missing or necessary for what
  you want to do. Let us know by personal email to the developers, or the
  mantle convection or \texttt{aspect-devel} mailing lists hosted at
  \url{http://lists.geodynamics.org/cgi-bin/mailman/listinfo/aspect-devel}!}

\subsection{Referencing \aspect{}}

As with all scientific work, funding agencies have a reasonable expectation
that if we ask for continued funding for this work, we need to demonstrate
relevance. 
In addition, many have contributed to the development of \aspect{} and deserve credit
for their work.
To this end, we ask that if you publish results that were obtained
to some part using \aspect{}, please see the information at
{\bf \url{https://aspect.geodynamics.org/cite.html}}, which includes suggestions
for acknowledgments and citations.

Also see \cite{aspect-doi-v1.5.0,aspect-doi-v2.0.0,aspectmanual,KHB12,heister_aspect_methods2}.

\subsection{Acknowledgments}

The development of \aspect{} has been funded
through a variety of grants to the authors. Most immediately, it has been
supported through the Computational Infrastructure in Geodynamics
grant, initially by the CIG-I grant (National Science Foundation Award No. EAR-0426271,
via The California Institute of Technology) and later by the CIG-II
and CIG-III grants
(National Science Foundation Awards No. EAR-0949446 and EAR-1550901, via The University
of California -- Davis). In addition, the libraries upon
which \aspect{} builds heavily have been supported through many other grants
that are equally gratefully acknowledged.

Please acknowledge CIG as follows:
{\parindent0pt
  \begin{center}
    \shadowbox{
      \begin{minipage}[c]{0.9\linewidth}
ASPECT is hosted by the Computational Infrastructure for Geodynamics (CIG)
which is supported by the National Science Foundation award EAR-1550901.
      \end{minipage}
    }
  \end{center}
}
