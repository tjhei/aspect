\section{Installation}
\label{sec:installation}

There are three distinct ways to install ASPECT -- compilation
from source, installing a virtual machine, and using a Docker container --
each providing distinct advantages and disadvantages. In this section we
describe all three options and start with a summary of their properties to
guide users to an informed decision about the best option for their purpose.

\begin{table}[htb]
  \center
  \begin{tabular}{|c|ccc|}
    \hline
    Feature & Compile \& Install & Virtual Machine & Docker Container \\
    \hline
    Speed overhead          & 0\%   & 30\%     & 0--5\%    \\
    Disk overhead           & 0~GB  & 1~GB     & 200~MB       \\
    Knowledge required      & Much  & Very Little & Little    \\
    Root privileges required & No   & No (installed VM software) & Partially  \\
    Embedded in native environment & Yes & No  & Partially    \\
    MacOS support           & Yes   & Yes      & Yes    \\
    Windows support         & No    & Yes      & Yes    \\
    Local parallelization   & Yes   & Yes      & Yes            \\ 
    Massively parallel computations & Yes & No & No \\ 
    Modifying ASPECT        & Possible & Possible & Possible \\
    Configuring dependencies & Possible & No   & No \\ \hline
  \end{tabular}
  \caption{Features of the different installation options of \aspect{}.}
  \label{tab:install-options}
\end{table}

The available options can be best presented in form of typical use cases:

\begin{enumerate}
\item Virtual Machine (\aspect{} beginner and tutorial participant): The
virtual machine image provides a fully prepared user environment that contains
installations of \aspect{}, all required libraries, and visualization software
on top of a full Linux environment. This way beginning users and tutorial
participants can work in a unified  environment, thus minimizing installation
time and technical problems. Due to the overhead of virtualizing a full
operating system this installation typically needs more space, and is
approximately 30~\% slower than a native installation. Additionally working in a
virtual machine `feels' differently from working in your usual desktop
environment. The virtual machine can be run on all host operating systems that
can run a virtualization software like VirtualBox (e.g. Linux, Apple MacOS,
Microsoft Windows).

\item Docker Container (advanced user with no need to configure/change the
underlying libraries, possibly changing parts of \aspect): Docker containers are
lightweight packages that only encapsulate the minimal dependencies to run an
application like \aspect{} on top of the host operating system. They allow easy
installation and usage of \aspect{} in a unified environment, while relying on
the user's operating system to provide visualization software and model input
data. When compared to the virtual machine it is simple to exchange files
between the host operating system and the docker container, and it provides the
benefit to work in the desktop environment you are used to. They have very
little overhead in terms of memory and speed compared to virtual machines, and
allow for reproducible computations. The container is set up with a standard
\aspect{} installation, but this can be modified by advanced users (source code
development within the container is possible).

\item Compile \& Install (advanced users and developers with the need to
reconfigure underlying libraries or running massively parallel models): The most
advanced option is to compile and install \aspect{} from source. This allows
maximal control over the underlying libraries like \trilinos{} and \dealii{}, as
well as easy modifications to \aspect{} by recompiling a modified source
directory. Our installation instructions cover most Linux and MacOS operating
systems, but incompatibilities on individual systems can always occur and make
the installation more cumbersome. If you are planning to run massively parallel
computations on a compute cluster this is likely your only option. Since
clusters usually have a very individual setup, it is always a good idea to ask
IT support staff for help when installing \aspect{}, to avoid hard to reproduce
setup problems, and performance penalties.
\end{enumerate}

\subsection{Docker Container}
\label{subsec:docker_container}

\subsubsection{Installing Docker and downloading the \aspect{} image}

Docker is a lightweight virtualization software that allows to ship
applications with all their dependencies in a simple way. It is outside of the
scope of this manual to explain all possible applications of Docker, and we
refer to the introduction (\url{https://www.docker.com/what-docker}) and
installation and quickstart guides
(\url{https://www.docker.com/products/docker}) on the Docker website for more
detailed descriptions of how to set up and use the docker engine. More
importantly Docker provides a marketplace for exchanging prepared docker images
(called Docker Hub). After setting up the docker engine downloading a
precompiled \aspect{} image from Docker Hub is as simple as typing in a
terminal:

\begin{lstlisting}[frame=single,language=ksh]
docker pull gassmoeller/aspect
\end{lstlisting}

Note that the transfer size of the compressed image containing \aspect{} and
all its dependencies is about 900~MB. When extracted the image requires about
3.2~GB of disk space.

\subsubsection{Running \aspect{} models}
Although it is possible to use the downloaded \aspect{} docker image in a
number of different ways, we recommend the following workflow:

\begin{enumerate}
\item Create your \aspect{} input file in a folder of your choice (possibly
also containing any input data that is required by your model) and navigate in a
terminal into that directory.
\item Run the docker image and mount the current directory as a read-only
volume into the docker container\footnote{Note that it is possible to mount a
directory as writeable into the container. However, this is often associated
with file permission conflicts between the host system and the container.
Therefore, we recommend this slightly more cumbersome, but also more reliable
workflow.}. This is accomplished by specifying the -v flag followed by
the absolute path on the host machine, colon, absolute path within the docker
container, colon, and specifying read-only permissions as in the example below.

Make sure your parameter file specifies a model output directory \textit{other}
than the input directory, e.g. \texttt{/home/dealii/aspect/model\_output}. When
you have started the container run the aspect model inside the container. Note
that there are two \aspect{} executables in the work directory of the container:
\texttt{aspect} and \texttt{aspect-release}. For a discussion of the
different versions see Section~\ref{sec:debug-mode}, in essence: You should run
\texttt{aspect} first to check your model for errors, then run
\texttt{aspect-release} for a faster model run.

To sum up, the steps you will want to execute are:
\begin{lstlisting}[frame=single,language=ksh,showstringspaces=false]
docker run -it -v "$(pwd):/home/dealii/aspect/model_input:ro" \
  gassmoeller/aspect:latest bash
\end{lstlisting}

Within the container, simply run your model by executing:

\begin{lstlisting}[frame=single,language=ksh]
./aspect model_input/your_input_file.prm
\end{lstlisting}

\item After the model has finished (or during the model run if you want to check
intermediate results) copy the model output out of the container into your
current directory. For this you need to find the name or ID of the docker
container by running \texttt{docker ps -a} in a separate terminal first. Look
for the most recently started container to identify your current \aspect{}
container.

Commands that copy the model output to the current directory could be:
\begin{lstlisting}[frame=single,language=ksh]
docker ps -a # Find the name of the running / recently closed container in the output
docker cp CONTAINER_NAME:/home/dealii/aspect/model_output .
\end{lstlisting}

\item The output data is saved inside your container even after the computation
finishes and even when you stop the container. After you have copied the data
out of the container you should therefore delete the container to avoid
duplication of output data. Even after deleting you will always be able to start
a new container from the downloaded image following step 2. Deleting the
finished container can be achieved by the \texttt{docker container prune}
command that removes any container that is not longer running.
\note{If you own other finished containers that you want to keep use
\texttt{docker container rm CONTAINER\_NAME} to only remove the container named
\texttt{CONTAINER\_NAME}.}

To remove all finished containers use the following command:
\begin{lstlisting}[frame=single,language=ksh]
docker container prune
\end{lstlisting}
Alternatively only remove a particular container:
\begin{lstlisting}[frame=single,language=ksh]
docker container rm CONTAINER_NAME
\end{lstlisting}
\end{enumerate}

You are all set. Repeat steps 1-4 of this process as necessary when updating
your model parameters. 

\subsubsection{Developing \aspect{} within a container}

The above given workflow does not include advice on how to modify \aspect{}
inside the container. We recommend a slightly different workflow for advanced
users that want to modify parts of \aspect{}. The \aspect{} docker container
itself is build on top of a \dealii{} container that contains all dependencies
for compiling \aspect{}. Therefore it is possible to run the deal.II container,
mount an \aspect{} source directory from your host system and compile it inside
of the container. An example workflow could look as following (assuming you
navigated in a terminal into the modified \aspect{} source folder):

\begin{lstlisting}[frame=single,language=ksh,showstringspaces=false]
docker pull dealii/dealii:v8.5.pre.4-gcc-mpi-fulldepsmanual-debugrelease
docker run -it -v "$(pwd):/home/dealii/aspect:ro" \
  dealii/dealii:v8.5.pre.4-gcc-mpi-fulldepsmanual-debugrelease bash
\end{lstlisting}

Inside of the container you now find a read-only \aspect{} directory that
contains your modified source code. You can compile and run a model inside the
container, e.g. in the following way:

\begin{lstlisting}[frame=single,language=ksh]
mkdir aspect-build
cd aspect-build
cmake -DCMAKE_BUILD_TYPE=Debug -DDEAL_II_DIR=$HOME/deal.II-install $HOME/aspect
./aspect $HOME/aspect/cookbooks/shell_simple_2d.prm
\end{lstlisting}

To avoid repeated recompilations of the \aspect{} source folder we recommend to
reuse the so prepared container instead of starting new containers based on the
\dealii{} image. This can be achieved by the following commands outside of the
container:

\begin{lstlisting}[frame=single,language=ksh]
docker ps -a # Find the name of the running / recently closed container in the output
docker restart CONTAINER_NAME
docker attach CONTAINER_NAME
\end{lstlisting}

For more information on the differences between using images and containers,
and how to attach additional terminals to a running container, we refer to the
docker documentation (e.g.
\url{https://docs.docker.com/engine/getstarted/step_two/}).

\subsection{Virtual Machine}

\subsubsection{Installing VM software and setting up the virtual machine}

The \aspect{} project provides an experimental virtual machine containing a
fully configured version of \aspect{}. To use this machine, you will need to
install VirtualBox (\url{http://www.virtualbox.org/}) on your machine, and then
import a virtual machine image that can be downloaded from
\url{http://www.math.clemson.edu/~heister/dealvm/}. Note, however, that the
machine image is several gigabytes in size and downloading will take a while.
After downloading and installing the virtual image it is convenient to set up a
shared folder between your host system and the virtual machine to exchange model
files and outputs.

\subsubsection{Running \aspect{} models}

The internal setup of the virtual machine is similar to the Docker container
discussed above, except that it contains a full-featured desktop environment.
Also note that the user name is \texttt{ubuntu}, not \texttt{dealii} as in the
Docker container. Again there are multiple ways to use the virtual machine, but
we recommend the following workflow:

\begin{enumerate}
\item Create your \aspect{} input file in the shared folder and start the
virtual machine.
\item Navigate in a terminal to your model directory.
\item Run your model using the provided \aspect{} executable:

\begin{lstlisting}[frame=single,language=ksh]
~/aspect/aspect your_input_file.prm
\end{lstlisting}

\item The model output should automatically appear on your host machine in the
shared directory.

\item After you have verified that your model setup is correct, you might want
to consider recompiling \aspect{} in release mode to increase the speed of the
computation. See Section~\ref{sec:debug-mode} for a discussion of debug and
release mode.

\item Visualize your model output either inside of the virtual machine
(ParaView and VisIt are pre-installed), or outside on your host system.
\end{enumerate}

You are all set. Repeat steps 1-6 of this process as necessary when updating
your model parameters. 

\subsection{Local installation}

This is a brief explanation of how to compile and install the required dependencies and
\aspect{} itself. This installation procedure guarantees fastest runtimes, and largest flexibility,
but usually requires more work than the options mentioned in the previous sections. 
While it is possible to install ASPECT's dependencies in particular \pfrst{}, \trilinos{}, 
and \dealii{} manually, we recommend to use the 
\texttt{candi} software (see \url{https://github.com/dealii/candi}). \texttt{candi} was written
as an installation program for deal.II, and includes a number of system specific instructions
that will be listed when starting the program. It can be flexibly configured to allow for
non-default compilers or libraries (e.g. Intel's MKL instead of LAPACK) by changing entries
in the configuration file \texttt{candi.cfg}, or by providing platform specific installation files.

In case you encounter problems during the installation, please consult our wiki
(\url{https://github.com/geodynamics/aspect/wiki}) for frequently asked
questions and special instructions for MacOS users, before posting your
questions on the mailing list.

\subsubsection{System prerequisites}

\texttt{candi} will show system specific instructions on startup, but its prerequisites
are relatively widely used and packaged
for most operating systems. You will need compilers for C, C++ and
Fortran, the GNU make system, the CMake build system, and the libraries and
header files of BLAS, LAPACK and zlib, which is used for compressing
the output data. To use more than one process for your computations
you will need to install a MPI library, its headers and the
necessary executables to run MPI programs. There are some optional packages
for additional features, like the HDF5 libraries for additional output formats,
\petsc{} for alternative solvers, and Numdiff for checking \aspect{}'s test
results with reasonable accuracy, but these are not strictly required, and in
some operating systems they are not available as packages but need to be
compiled from scratch.
Finally, for obtaining a recent development version of \aspect{} you will
need the git version control system.

An exemplary command to obtain all required packages on Ubuntu 14.04 would be:
\begin{verbatim}
sudo apt-get install build-essential \
                     cmake \
                     gcc \
                     g++ \
                     gfortran \
                     git \
                     libblas-dev \
                     liblapack-dev \
                     libopenmpi-dev \
                     numdiff \
                     openmpi-bin \
                     zlib1g-dev
\end{verbatim}

\subsubsection{Using candi to compile dependencies}

In its default configuration \texttt{candi} downloads and
compiles a \dealii{} configuration that is able to run \aspect, but it
also contains a number of packages that are not required (and that can
be safely disabled if problems occur during the
installation). We require at least the packages \pfrst{}, \trilinos{} 
(or as an experimental alternative \petsc{}), and finally \dealii{}.
 
At the time of this writing \texttt{candi} will install \pfrst{} 2.0,
\trilinos{} 12.10.1,  \petsc{} 3.6.4, and \dealii{} 8.5.0. 
We strive to keep the development version of \aspect{} compatible with 
the latest release of \dealii{} and the current \dealii{} development 
version at any time, and we usually support several older versions of
\pfrst{}, \trilinos{}, and \petsc{}.

\begin{enumerate}
\item \textit{Obtaining candi:} Download \texttt{candi} by running 
    \begin{verbatim}
    git clone https://github.com/dealii/candi
    \end{verbatim} 
    in a directory of your choice. 

\item \textit{Installing \dealii{} and its dependencies:} Execute \texttt {candi} by running
    \begin{verbatim}
    cd candi
    ./candi.sh -p INSTALL_PATH
    \end{verbatim} 
    (here we assume you replace \texttt{INSTALL\_PATH} by the path were
    you want to install all dependencies and \dealii{}, typically a directory inside
    \texttt{\$HOME/bin} or a similar place). 
    This step might take a long time, but can be parallelized by adding 
    \texttt{-jN}, where 
    \texttt{N} is the number of CPU cores available on your computer. Further configuration options 
    and parameters are listed at \url{https://github.com/dealii/candi}.

\item You may now want to configure your environment to make it aware of the newly installed
    packages. This can be achieved by adding the line 
    \texttt{source INSTALL\_PATH/configuration/enable.sh} to the file responsible for setting
    up your shell environment\footnote{For bash this would be the file \texttt{\~{}/.bashrc}.} 
    (again we assume you replace \texttt{INSTALL\_PATH} by the patch chosen in the previous step).
    Then close the terminal and open it again to activate the change.

\item \textit{Testing your installation:} Test that your installation works
  by compiling the {\texttt{step-32}} example that you can find in
  {\texttt{\$DEAL\_II\_DIR/examples/step-32}}. Prepare and compile by running {\texttt{cmake . \&\& make}} 
  and run with {\texttt{mpirun -n 2 ./step-32}}.

\end{enumerate}

Congratulations, you are now set up for compiling \aspect{} itself.

\subsubsection{Obtaining \aspect{} and initial configuration}

The development version of \aspect{} can be downloaded by executing the command
\begin{verbatim}
 git clone https://github.com/geodynamics/aspect.git
\end{verbatim}
If {\texttt{\$DEAL\_II\_DIR}} points to your \dealii{} installation, you can configure
\aspect{} by running
\begin{verbatim}
 cmake .
\end{verbatim}
in the \aspect{} directory created by the {\texttt{git clone}} command above.
If you did not set {\texttt{\$DEAL\_II\_DIR}} you have to supply cmake with the location:
\begin{verbatim}
 cmake -DDEAL_II_DIR=/u/username/deal-installed/ .
\end{verbatim}

An alternative would be to configure \aspect{} as an out-of-source build. 
You would need to create a separate build directory and
specify \aspect{}'s source directory using \texttt{cmake PATH\_TO\_ASPECT\_SOURCE}
from within the build directory. The
instructions in the following sections assume an in-source build.

\subsubsection{Compiling \aspect{} and generating documentation}
\label{sec:compiling}

After downloading \aspect{} and having built the libraries it builds on, you
can compile it by typing
\begin{verbatim}
  make
\end{verbatim}
on the command line (or \texttt{make -jN} if you have multiple processors in
your machine, where \texttt{N} is the number of processors). This builds the
\aspect{} executable which will reside in
the main directory and will be named \texttt{./aspect}. If you intend to
modify \aspect{} for your own experiments, you may want to also generate
documentation about the source code. This can be done using the command
\begin{verbatim}
  cd doc; make
\end{verbatim}
which assumes that you have the \texttt{doxygen} documentation generation tool
installed. Most Linux distributions have packages for \texttt{doxygen}. The
result will be the file \url{doc/doxygen/index.html} that is the starting
point for exploring the documentation.
